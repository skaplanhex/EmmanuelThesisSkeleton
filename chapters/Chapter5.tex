%%%%%%%%%%%%% Event Selection %%%%%%%%%%%%%%%%%%%%
\chapter{Physics Object and Event Selections}
\label{chap:PhysicsObjectEventSelections}


%%%%%%%%%%%%% Objection identification %%%%%%%%%%%%%%%%%%%%
\section{Objection identification and selection}
\label{sec:ObjectionIdentificationSelection}


%%%%%%%%%%%%% Muon selections %%%%%%%%%%%%%%%%%%%%
\subsection{Muon selections}
\label{sec:MuonSelections}


We briefly describe the meaning of the different selection criteria below.
\begin{itemize}
\item We require PF muons with $\pt > 10\GeV$ and $\abs{\eta} < 2.4$.
\item Global $\chi^{2}/\textrm{ndof}$ is the $\chi^{2}$ per degree of freedom of the 
fit to the tracks left behind by the muon as it transverses the detector.
\item \textbf{\textit{TrackerLayersWithMeasurement}}: is the number of tracker layers with hits.
\end{itemize}


\begin{table*}[!hbtp]
\centering
\caption{Muon selection requirements.}
\label{tab:MuonSelections}
\begin{tabular}{l|c}
\hline \hline
Muon observable & Selection \\
\hline \hline
Identification~(ID) & Particle Flow and Global \\
Transverse momentum~$\pt$ & $> 10\GeV$ \\
Pseudo rapidity~$\abs{\eta}$ & $< 2.4$ \\
Global $\chi^{2}/\textrm{ndof}$ of the fit & $< 10$  \\
Number of valid pixel hits & $>$ 0 \\
TrackerLayersWithMeasurement & $>$5 \\
Number of valid hits in muon chamber & $> 0$ \\
Number of muon stations with muon segments & $> 1$ \\
Transverse impact parameter~$\abs{d_{xy}(PV)}$ & $< 0.02\unit{cm}$ \\
Longitudinal impact parameter~$\abs{d_{z}(PV)}$ & $< 0.5\unit{cm}$ \\
Relative PF isolation~$I_{\textrm{rel}}$ within $\Delta{R} < 0.3$, & \\
with beta corrections for pileup & $< 0.15$ \\
\hline \hline
\end{tabular}
\end{table*}


%%%%%%%%%%%%% Electron selections %%%%%%%%%%%%%%%%%%%%
\subsection{Electron selections}
\label{sec:ElectronSelections}

 
We briefly describe the meaning of the different selection criteria below.
\begin{itemize}
\item We require PF electrons with $\pt > 10\GeV$ and $\abs{\eta} < 2.4$, but with the 
additional requirement that electrons in the gap region $1.4442 < \abs{\eta} < 1.566$ 
are rejected.
\item $\Delta{\eta_{\textrm{In}}}$ is the pseudorapidity difference between SuperCluster position 
and track direction at vertex extrapolated to ECAL assuming no radiation.
\item $\Delta{\Phi_{\textrm{In}}}$ is the azimuthal difference between SuperCluster position 
and track direction at vertex extrapolated to ECAL assuming no radiation.
\item $\sigma_{i \eta i \eta}$ is the supercluster $\eta$ width taken from the covariance matrix 
using logarithmic weights, where $i \eta$ represents the $i^{th}$ ECAL crystal in 
the $\eta$ direction. In other words, it is the second moment of the ECAL energy cluster 
distribution in the $\eta$ direction.
\item Hadronic leakage variable $H/E$ is the ratio between the energy deposit recorded in 
the HCAL tower, just behind the ECAL supercluster seed, and the ECAL supercluster
energy associated with the electron.
\item $\abs{1/E -1/p}$ is the absolute value of the difference between the reciprocal of the 
electron energy and the reciprocal of the magnitude of the electron momentum.
\item Conversion rejection cut allows for rejecting electrons identified as originating from 
the conversion of a photon. Further information on photon conversion is found 
in Section~\ref{sec:AsymmeticInternalPhotonConversionFakeRate}.
\item Electrons are required to be $\Delta{R} > 0.1$ from selected muons.
\end{itemize}


\begin{table*}[!hbtp]
\centering
\caption{Electron selection requirements. Several electron ID criteria are different 
for the barrel ($\abs{\eta} < 1.44$) and endcap ($1.56 < \abs{\eta} < 2.4$) regions.
Electrons in the gap region $1.4442 < \abs{\eta} < 1.566$ are rejected.}
\label{tab:ElectronSelections}
\begin{tabular}{l|cc}
\hline \hline
\multirow{2}{*}{Electron observable} & \multicolumn{2}{c} {Selection} \\
& Barrel & Endcap \\
\hline \hline
Identification~(ID) & Particle Flow &  Particle Flow \\
Transverse momentum~$\pt$ & $> 10\GeV$ & $> 10\GeV$ \\
Pseudo rapidity~$\abs{\eta}$ & $< 2.4$ & $< 2.4$ \\
Spatial ($\eta$) matching between track and supercluster~$\Delta{\eta_{\textrm{In}}}$ & $< 0.007$ & $< 0.009$ \\
Spatial ($\Phi$) matching between track and supercluster~$\Delta{\Phi_{\textrm{In}}}$ & $< 0.15$ & $< 0.10$ \\
Transverse shape of the electromagnetic cluster~$\sigma_{i \eta i \eta}$ & $< 0.01$ & $< 0.03$ \\
Hadronic leakage variable~$H/E$ & $< 0.12$ & $< 0.10$ \\
Transverse impact parameter~$\abs{d_{xy}(PV)}$ & $< 0.02$ & $< 0.02\unit{cm}$ \\
Longitudinal impact parameter~$\abs{d_{z}(PV)}$ & $< 0.1$ & $< 0.2\unit{cm}$ \\
$\abs{1/E -1/p}$ & $< 0.05$ & $< 0.05$ \\
Relative PF isolation $I_{\textrm{rel}}$ within~$\Delta{R} < 0.3$ & $< 0.15$ & $< 0.15$ \\ 
Conversion rejection cut & 0 & 0 \\
Number of missing expected inner tracker layer hits & $< 2$ & $< 2$ \\
$\Delta{R}$ to nearest selected muon & $> 0.1$ & $> 0.1$ \\
\hline \hline
\end{tabular}
\end{table*}


%%%%%%%%%%%%% Photon selections %%%%%%%%%%%%%%%%%%%%
\subsection{\texorpdfstring{$\tauh$}{Hadronic tau}-lepton selections}
\label{sec:HadronicTauSelections}


We briefly describe the meaning of the different selection criteria below.
\begin{itemize}
\item We require HPS PF Taus with $\pt > 20\GeV$ and $\abs{\eta} < 2.3$.
\item \textbf{\textit{ByDecayModeFinding}}: A discriminant value that determines whether the HPS algorithm 
is able to reconstruct one and three-prong decay modes for $\tauh$-lepton candidates.
\item \textbf{\textit{AgainstElectronLoose}}: A discriminant value that rejects misidentified electrons, since 
they are prone to be reconstructed as one-prong $\tauh$-leptons, by requiring  
that the electron pion multivariate analysis (MVA) value is less than 0.6. 
\item \textbf{\textit{AgainstMuonLoose}}: A discriminant that rejects misidentified muons by requiring
that there is no $\Delta R$ matching between the leading track of the $\tauh$-lepton 
candidate and chamber hits left by a muon.
\item \textbf{\textit{ByLooseCombinedIsolationDBSumPtCorr}}: An isolation discriminant value corresponding 
to the sum $\pt$ of all charged and neutral candidates in the isolation annulus, taking into account the effects of pileup, with $\pt$ greater than $0.5\GeV$ and $\Delta R = 0.5$.
\item $\tauh$-leptons are required to be $\Delta{R} > 0.1$ from selected light-leptons.
\end{itemize}


\begin{table*}[!hbtp]
\centering
\caption{$\tauh$-lepton selection requirements.}
\label{tab:HadronicTauSelections}
\begin{tabular}{l|c}
\hline \hline
Hadronic $\tau$-lepton observable & Selection \\
\hline \hline
Identification (ID) & HPS Particle Flow \\
Transverse momentum~$\pt$ & $> 20\GeV$ \\
Pseudo rapidity~$\abs{\eta}$ & $< 2.3$ \\
ByDecayModeFinding & 1  \\
AgainstElectronLoose & 1 \\
AgainstMuonLoose & 1 \\
ByLooseCombinedIsolationDeltaBetaCorr & 1 \\
PF isolation~$E^{\textrm{iso}}_{\tau}$ & $< 2.0\GeV$ \\
$\Delta{R}$ to nearest selected light-leptons & $> 0.1$ \\
\hline \hline
\end{tabular}
\end{table*}


%%%%%%%%%%%%% Photon selections %%%%%%%%%%%%%%%%%%%%
\subsection{Photon selections}
\label{sec:PhotonSelections}


A summary of all the photon selections is given in Table~\ref{tab:PhotonSelections}.

\begin{table*}[!hbtp]
\centering
\caption{Photon selection requirements. Several photon ID criteria are different 
for the barrel ($\abs{\eta} < 1.44$) and endcap ($1.56 < \abs{\eta} < 2.4$) regions.}
\label{tab:PhotonSelections}
\begin{tabular}{l|cc}
\hline \hline
\multirow{2}{*}{Photon observable} & \multicolumn{2}{c} {Selection} \\
& Barrel & Endcap \\
\hline \hline
Transverse momentum~$\pt$ & $> 10\GeV$ & $> 10\GeV$ \\
Pseudo rapidity~$\abs{\eta}$ & $< 2.4$ & $< 2.4$ \\
Conversion safe electron veto &1&1 \\
Single tower H/E & $< 0.06$ & $< 0.05$ \\
Transverse shape of the electromagnetic cluster~$\sigma_{i \eta i \eta}$ & $< 0.011$ & $< 0.034$ \\
Rho corrected relative PF charged hadron isolation & $< 0.06$ & $< 0.05$ \\
Rho corrected relative PF neutral hadron isolation & $<$ 0.16 & $< 0.10$ \\
Rho corrected PF photon isolation & $< 0.08$ & $< 0.12$ \\
\hline \hline
\end{tabular}
\end{table*}


%%%%%%%%%%%%% Jet selections %%%%%%%%%%%%%%%%%%%%
\subsection{Jet selections}
\label{sec:JetSelections}


We briefly describe the meaning of the different selection criteria below.
\begin{itemize}
\item We require PF jets with $\pt > 30\GeV$ and $\abs{\eta} < 2.5$.
\item \textbf{\textit{Number of constituents}}: A jet must consist of more than one constituent in its reconstruction. 
\item \textbf{\textit{Residual corrections}}: For simulation samples we apply L1FastL2L3 corrections to 
the PF jets, while for data we apply L1FastL2L3residual corrections. The $\pt$ and $\eta$ 
dependence between simulation and data differ, therefore, a correction is applied to jets 
in data in order to remove the observed difference.
\item \textbf{\textit{Neutral hadron fraction}}: The fraction of energy deposited in the hadronic calorimeter 
from neutral particles is required to be less than 0.99. 
\item \textbf{\textit{Neutral EM fraction}}: The fraction of energy deposited in the electromagnetic calorimeter 
from neutral particles is required to be less than 0.99.  
\item \textbf{\textit{Charged hadron fraction}}: The fraction of energy deposited in the hadronic calorimeter 
from charged particles must be greater than zero for $\abs{\eta} < 2.4$.
\item \textbf{\textit{Charged EM fraction}}: The fraction of energy deposited in the electromagnetic calorimeter 
from charged particles must be smaller than 0.99 for $\abs{\eta} < 2.4$.
\end{itemize}


\begin{table*}[!hbtp]
\centering
\caption{Jet selection requirements.}
\begin{tabular}{l|cc}
\hline \hline
\multirow{1}{*}{Jet observable} & \multicolumn{1}{c} {Selection} \\
\hline \hline
Transverse momentum~$\pt$ & $> 30\GeV$ \\
Pseudo rapidity~$\abs{\eta}$ & $< 2.5$ \\
Number of constituents in jet &  $> 1$  \\
Neutral hadron fraction of total jet energy &  $< 0.99$  \\
Neutral EM fraction of total jet energy & $< 0.99$ \\
Charged hadron fraction~$(\abs{\eta} < 2.4)$ & $> 0$ \\
Charged EM fraction~$(\abs{\eta} < 2.4)$ & $< 0.99$ \\
Number of tracks~$(\abs{\eta} < 2.4$) & $> 0$ \\
\hline \hline
\end{tabular}
\end{table*}


%%%%%%%%%%%%% Jet selections %%%%%%%%%%%%%%%%%%%%
\subsection{Tagged \texorpdfstring{$b$}{b}-jet selections}
\label{sec:TaggedBJetSelections}


%%%%%%%%%%%%% Jet selections %%%%%%%%%%%%%%%%%%%%
\subsection{\texorpdfstring{\MET}{Missing transverse energy} selection}
\label{sec:METSelections}


The quality selection requirements for $\MET$ is given by:
\begin{itemize}
\item We use PFMET physics objects.
\item The following filters, recommended by the CMS collaboration, are 
applied to $\MET$:
\begin{itemize}
\item \textbf{\textit{CSC tight beam halo filter}}: Secondary particle showers resulting 
from beam-gas collisions in the vacuum chambers induce beam halo noise in the detector. 
Therefore, this filter is used in order to identify events containing large beam backgrounds. 
\item \textbf{\textit{HBHE noise filter}}: Rejects isolated noise originating from the HCAL 
barrel and endcap readout electronics, which may be mis-reconstructed as hadronic 
energy deposit.
\item \textbf{\textit{ECAL dead cell trigger primitive filter}}: This is applied in order to 
reject fake $\MET$ coming from high energy particles which have deposited their energy 
in noisy crystal cells in the ECAL that have been left out of the event reconstruction. 
\item \textbf{\textit{Tracking failure filter}}: The tracking algorithms may fail for some of its 
iterations when too many clusters hits are found. Therefore, a selection requirement, based 
on the ratio of the $\sum \pt$ of all tracks associated with the primary vertex to $\HT$ in the 
event, is applied to reject such events. 
\item \textbf{\textit{Bad EE Supercrystal filter}}: Designed to reject events containing 
anomalously large energy in the ECAL endcap superclusters. 
\item \textbf{\textit{Primary vertex filter}}: At least one ``good" primary vertex is required 
to be reconstructed in the event. A PV is identified as good if it has a number of degree of 
freedom $n_{dof} > 4$ and has a position of $\abs{z} < 24$ \unit{cm} and $\rho < 2$ \unit{cm}, 
to reject noisy events, due to pileup, and ensure good collision candidates. 
\end{itemize}
\end{itemize}


%%%%%%%%%%%%% Trigger efficiency %%%%%%%%%%%%%%%%%%%%
\section{Measurement of the lepton trigger efficiency}
\label{sec:LeptonTriggerEfficiency}


%%%%%%%%%%%%% Lepton isolation and identification efficiency %%%%%%%%%%%
\section{Measurement of the lepton isolation and identification efficiency}
\label{sec:LeptonEfficiency}



%%%%%%%%%%%%% Applying B-jet weights and Systematics %%%%%%%%%%%%%%%%
\subsection*{\texorpdfstring{$\cPqb$}{b}-tagging scale factor}
\label{sec:BTaggingSF}


For $\cPqb$-quark jets that are tagged as a $\cPqb$-jet, the scale factor for the event 
is given by,
\begin{flalign}
SF = \prod^{N_{\textrm{jet}}}_{i=1} w^{\textrm{b-tagged}}_{i}, \textrm{~where~} w^{\textrm{b-tagged}}_{i} = \frac{\epsilon^{\textrm{Data}}_{i}}{\epsilon^{\textrm{MC}}_{i}},
\end{flalign}
whereas, for $\cPqb$-quark jets that are not tagged as a $\cPqb$-jet, the scale factor is given by,
\begin{flalign}
SF = \prod^{N_{\textrm{jet}}}_{i=1} w^{\textrm{not b-tagged}}_{i}, \textrm{~where~} w^{\textrm{not b-tagged}}_{i} = \frac{1 - \epsilon^{\textrm{Data}}_{i}}{1 - \epsilon^{\textrm{MC}}_{i}} = \frac{1 - w^{\textrm{b-tagged}}_{i} \cdot \epsilon^{\textrm{MC}}_{i}}{1 - \epsilon^{\textrm{MC}}_{i}}
\end{flalign}
where, $w^{\textrm{b-tagged}}$ and $\epsilon^{\textrm{MC}}$ are supplied by CMS. If an event has 
both $\cPqb$-tagged and non-$\cPqb$-tagged jets then the overall scale factor is the product of the two 
above expressions. Systematic uncertainties related to the $\cPqb$-tagging scale factors are described 
in Section~\ref{sec:BTaggingSFUncertainty}.


%%%%%%%%%%%%% Event selection %%%%%%%%%%%%%%%%%%%%
\section{Event selection}
\label{sec:EventSelection}